\label{sec:easy_algorithms}
%-----------------------------------------------------------------------


\subsection{Lineare Interpolation}

Bei der Verwendung einer linearen Interpolation werden meistens Fälle betrachtet, in denen sich Lücken zwischen bereits vorhandenen Daten befinden. Dabei wird unterstellt, dass der Verbrauchsverlauf nahezu linear ist. Für jeden Verbrauchszeitpunkt $x$, dessen Verbrauchswert $y$ unbekannt ist, eine Gerade zwischen zwei benachbarte Punkte $x_1$ und $x_2$ gelegt, wobei gilt $x1\leq\ $x $\leq \ x_2$. Für jeden unbekannten Datenpunkt $P(x/y)$, der auf dieser Gerade liegt, ergibt sich die Formel:
$$y = y_1 + \frac{y_2 - y_1}{x_2 - x_1} \cdot (x - x_1)$$
Die lineare Interpolation eignet sich dazu, zu ermitteln, in welchem Wertebereich sich die nicht erfassten Verbrauchswerte befinden können. Die Ergebnisse wurden in diesem Projekt als Vergleichswerte für die verbesserten Interpolations-Algorithmen verwendet, um deren Stabilität und Zuverlässigkeit zu bewerten. Für eine Erfassung der tatsächlichen Verbrauchswerte eignet sich die lineare Interpolation weniger, da zwar ein steter, aber kein glatter Verlauf der Funktion erzeugt wird. Darüber hinaus birgt sie die Problematik, dass sie nur die direkten benachbarten Verbrauchswerte betrachtet und Schwankungen im Verbrauchsverlauf ignoriert.
%-----------------------------------------------------------------------

\subsection{Polynomielle Interpolationsverfahren}
Die fehlenden Verbrauchswerte sollen im Folgenden auf Basis eines mathematischen Verfahrens ermittelt werden, das nicht nur die direkten benachbarten Punkte, sondern den gesamten Verlauf inklusive Schwankungen und Peaks berücksichtigt. Das Ziel ist es, einen möglichst realistischen Energieverbrauch zu rekonstruieren.\\
Hierfür werden polynomielle Interpolationsverfahren verwendet. Es wird ein Polynom gesucht, das exakt durch die bekannten Punkte aus den Messdaten verläuft. Durch die gegebenen Messdaten wird eine Kurve gelegt, um zu $n+1$ verschiedenen Datenpunkten ein Interpolationspolynom maximal $n-ten$ Grades zu finden.

\subsubsection{Interpolation nach Newton}
Für n Stützstellen wird eine Annäherung an die Polynomfunktion vom möglichst kleinen Grad an (bis maximal Grad $n$) durchgeführt. Gesucht ist eine Polynomfunktion für $n+1$ Stützstellen, mit der die fehlenden Verbrauchsdaten beim Zeitpunkt $n+1$ berechnet werden können. Das Interpolationspolynom ist folgendermaßen aufgebaut:
$$p_n(x)=a_0+a_1(x_n-x_0)+...+a_n(x_n-x_0)...(x_n-x_{n-1})$$
Für jeden weiteren fehlenden Messwert zum Zeitpunkt $n+1$ wird die Formel um $a_{n+1}(x_{n+1}-x_{0})...(x_{n+1}-x_{n})$ erweitert. Das hat den Vorteil, dass das Polynom bei nachfolgenden Punkten nicht komplett neu berechnet werden muss und somit Laufzeit eingespart wird, wohingegen vergleichbare Verfahren wie die Interpolation nach Lagrange hingegen das Polynom mit jedem fehlenden Datenpunkt neu aufbauen.\\
Allerdings hat die Interpolation nach Newton deutliche Schwächen: Das Polynom wird mit zunehmendem Grad instabiler und schwingt stark zwischen den Datenpunkten. Dadurch werden unrealistisch hohe oder niedrige Werte erzeugt, die in einem realen Energieverbrauchsverlauf niemals auftreten können. Deshalb sollte auf ein Polynom hohen Grades verzichtet werden.\\
Das nächste Interpolationsverfahren, das im Rahmen des Projektes implementiert wurde, betrachtet deshalb statt des gesamten Verlaufes nur die $n$ nächsten Nachbarn.

\subsubsection{Splines}
Bei der Spline-Interpolation wird für jeweils einen Teil der Daten ein geeignetes Polynom niedrigen Grades berechnet. Die vorhandenen Daten dienen als Stützstellen, zwischen denen fehlende Datenpunkte interpoliert werden. Für die Implementierung wurde ein Spline-Algorithmus gewählt, der auf die Messdaten eine Parabel $5-ten$ Grades mit den Koeffizienten $a_{i}x^5 + b_{i}x^4 + c_{i}x^3 + d_{i}x^2 + e_{i}x + f_{i}$ anwendet. Die Interpolation wurde dabei stückweise vorgenommen: Für jeden fehlenden Datensatz wurde aus den jeweils direkt benachbarten Messwerten ein eigenes Polynom berechnet.\\
Das Verfahren hat nicht nur den Vorteil, dass die Polynome nicht zunehmend instabiler werden, sondern es hat auch einen geringeren Rechenaufwand als das Newton-Verfahren, weil das Polynom nicht aus dem gesamten Datensatz berechnet wird. Da es allerdings nur zwischen vorhandenen Stützstellen interpoliert, eignet es sich nicht zur Vorhersage von Verbrauchswerte in der Zukunft.

%-----------------------------------------------------------------------


\subsection{\glqq Intelligente\grqq \,Interpolationsverfahren}
Für das Projekt ist es von hoher Bedeutung, möglichst realitätsnahe Werte zu prognostizieren, um eine Aussage über die Rentabilität der Solaranlage zu treffen. Deshalb hat das Projektteam die Vorgehensweise bei der Interpolation nicht nur aus rein mathematischer Perspektive, sondern auch auf Basis von Hintergrundwissen betrachtet. Der Energieverbrauch eines Haushaltes unterliegt keinem Funktionsgraphen oder anderen mathematischen Systematiken. Darüber hinaus lassen mathematische Interpolationsverfahren auch unrealistisch hohe oder niedrige Werte zu, die in der Praxis aber nie erreicht werden können. Aus diesem Grunde kam bei der Implementierung die Frage auf, wie sehr sich eine rein mathematische Vorgehensweise für die Aufgabenstellung überhaupt eigne.Das Projektteam entschied sich dazu, Algorithmen zu entwickeln, die nicht nur real vorhandene Daten stärker in die Interpolation mit einbeziehen, sondern die auch dieses Hintergrundwissen anwenden.
%-----------------------------------------------------------------------

\subsubsection{Interpolation aus Daten des Vortages}
Der Tagesablauf eines Haushaltes unterliegt meistens einem bestimmten Schema. Eine Person, die an Wochentagen tagsüber arbeiten geht und deshalb außer Haus ist, benutzt in dieser Zeitspanne keine Haushaltsgeräte. Strom wird lediglich von Geräten, die im Standby-Betrieb sind, verbraucht. Abends, in der Zeit zwischen Feierabend und Nachruhe, steigt der Energieverbrauch der Person, da sie u.a. den Kühlschrank öffnet, Essen kocht, den Computer bedient oder Licht einschaltet. An Wochentagen, d.h. von Montag bis Freitag, ist der Energieverbrauch deshalb eher gleichbleibend. Am Wochenende und an Feiertagen ist es hingegen realistisch, dass die Person auch tagsüber zu Hause ist und die Haushaltsgeräte bedient.\\
Das vorgestellte Interpolationsverfahren nutzt das Wissen, dass sich der Energieverbrauch aufgrund des Tagesrhythmus nicht stark ändert. Bei der Interpolation fehlender Verbrauchswerte wird der Wert des Vortages zur gleichen Uhrzeit verwendet, wobei hier das Verhältnis zwischen Wochentagen und Wochenenden/Feiertagen berücksichtigt wird: Fehlt an einem Montag ein Messwert, werden der vorherige Sonntag und Samstag übersprungen und es wird der Wert vom Freitag ausgewählt.\\
Die Ergebnisse des Algorithmus orientieren sich nahe am realen Verbrauchsverhalten. Jedoch hat er auch Schwächen: Den Personen, die im Haushalt leben, wird ein geregelter Tagesablauf unterstellt. Darüber hinaus können versehentlich Peaks erfasst werden, bspw. wenn sich die Person am vorherigen Tag schon früher zu Hause befand und während des betrachteten Zeitpunktes viele Geräte benutzt hat. Starke Verbrauchsschwankungen können das Ergebnis somit negativ beeinflussen. Der folgende Algorithmus berücksichtigt dieses Fehlerpotential und stellt deshalb eine Verbesserung des Verfahrens dar.

%-----------------------------------------------------------------------
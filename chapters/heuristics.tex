\section{Heuristiken}
Bei der Interpolation von Datenlücken kann es vorkommen, dass die prognostizierten Verbrauchswerte ungewöhnlich hoch oder niedrig sind. Das Risiko kann sogar bei der Einberechnung der einer breiteren Nachbarschaft von $n$ Werten entstehen, denn auch hier besteht die Möglichkeit, dass aus einem oder mehreren Peaks ein Interpolationswert gebildet wird. Deshalb werden die ermittelten Werte nach der Interpolation erneut durchlaufen, in Form von Heuristiken evaluiert und bei Bedarf nachträglich angepasst.

\subsection{Feiertage}
Der Energieverbrauch an Feiertagen gestaltet sich normalerweise anders als der Energieverbrauch an Wochentagen. Die Person ist an Feiertagen normalerweise zu Hause, weil sie nicht arbeiten muss und weil die Geschäfte geschlossen haben. Für die Verbrauchsschätzung werden die Feiertage deshalb wie Wochenend-Tage bewertet.\\
Die Messzeitpunkte werden im Datentyp LocalDateTime abgebildet. Dieser Datentyp kann zwar Wochentage erfassen, aber keine Feiertage. Des Weiteren gibt es für Java keine geeigneten Libraries, um Feiertage abhängig vom Wohnort (Land und Bundesland) zu konfigurieren. Das Projektteam hat deshalb ein Verfahren implementiert, das das Einlesen von iCalendar-Dateien zulässt, die lediglich im Eingabeordner abgelegt werden müssen.

\subsection{Mustererkennung aus Verbrauchsverläufen}
Wenn die Messdaten größtenteils vollständig und die Datenlücken überschaubar sind, ist es möglich, anhand der vorhandenen Daten ein Verbrauchsmuster zu erkennen. Das Verbrauchsmuster kann dann als Schablone dienen, die nachträglich auf die interpolierten Werte gelegt wird, um Datenausreißer zu identifizieren und auf einen realistischen Wert abzuändern.\\
Für diese Heuristik wurden im ersten Schritt Intervalle gebildet, die einen bestimmten Zeitabschnitt repräsentieren. Die Größe der Intervalle kann dabei individuell konfiguriert werden. Eine Möglichkeit ist die Bildung von Intervallen mit der Größe 4, d.h. [00:00-03:39] Uhr, [04:00-07:59] Uhr und folgende. Die Messwerte werden komplett durchlaufen und den Intervallen zugeordnet. Danach werden sowohl ein globaler Durchschnittsverbrauch für den gesamten Datensatz als auch der Durchschnittsverbrauch für jedes Intervall gebildet. Diese Kennzahl gibt Auskunft darüber, wie hoch der Stromverbrauch der Person im aktuellen Intervall ist. Liegt der Durchschnittsverbrauch eines Invervalls wie bspw. [00:00-03:59] Uhr deutlich unter dem globalen Durchschnittsverbrauch, ist es wahrscheinlich, dass die Person innerhalb des betrachteten Zeitrahmens normalerweise schläft oder außer Haus ist.\\
Anschließend wird diese Heuristik auf die interpolierten Werte angewendet. Für jedes Intervall können Toleranzbereiche konfiguriert werden, die vorgeben, ab welcher Abweichung vom Durchschnittsverbrauch der interpolierte Wert angepasst wird. Liegt der interpolierte Wert zum Beispiel mehr als $50\%$ über dem Durchschnitt, wird er auf den Durchschnittswert angepasst.


\subsection{Geeignete Verfahren für sehr große Datenlücken}
In manchen Fällen ist es nicht möglich, realistische Verbrauchsdaten aus dem Datensatz zu schätzen. So kann es vorkommen, dass eine Person nur Messdaten aus 2 oder 3 Monaten vorliegen hat, sich aber dennoch eine Interpolation für ein gesamtes Jahr wünscht, oder dass die gemessenen Verbrauchswerte sehr weit auseinander liegen. Falls der Energieverbrauch im erfassten Zeitbereich zusätzlich ungewöhnlich hoch oder niedrig war (z.B. weil die Person für einige Wochen im Urlaub war), kann dies die interpolierten Daten sehr verzerren.

\subsubsection{Anwendung von Statistiken}
Damit es dennoch möglich ist, einen annähernd realistischen Verbrauchsverlauf für den Haushalt zu erstellen, wird eine Heuristik angewendet, die sich auf offizielle Daten des Bundesverbandes der Energie- und Wasserwirtschaft (DEW) stützt. Diese Daten bilden sowohl den statistisch gemessenen Gesamtverbrauch eines Haushaltes abhängig von der Personenzahl als auch den prozentualen Anteil der jeweiligen Haushaltsgeräte am Gesamtverbrauch ab.\\
Die Heuristik wird nun ähnlich wie die Mustererkennung angewendet, die im vorherigen Abschnitt vorgestellt wurde. Ausgangsbasis für die Vergleiche sind allerdings nicht die Intervalldurchschnitte des vorhandenen Datensatzes, sondern die Durchschnitte, die sich aus der Statistik des BDEW berechnen lassen. Dafür werden künstliche Intervalle gebildet, die ebenfalls unter Verwendung von statistischen Daten die Ruhezeiten und die verbrauchsintensiven Tageszeiten repräsentieren. Die Intervallgrenzen werden auf Basis einer Datenerhebung von der Techniker Krankenkasse festgelegt, in der festgestellt wurde, zu welchen Uhrzeiten die Testpersonen aus statistischer Perspektive u.a. schlafen oder aufstehen.

\subsubsection{Saisonaler Verbrauch}
Auch dieses Verfahren dient dazu, einen Verbrauchsverlauf für sehr große Datenlücken zu konstruieren. Es ist dafür geeignet, direkt im Anschluss an die Mustererkennung mit statistischen Verbrauchsdaten auf den Datensatz angewendet zu werden.\\
Haushalte in den nördlichen Breitengraden haben im Winter einen höheren Energieverbrauch als im Sommer. Heizung und Beleuchtung werden regelmäßiger benutzt. In dieser Heuristik werden die interpolierten Werte nachträglich modifiziert und an die Jahreszeiten angepasst. Dies kann zum Einen auf Basis eines statistisch erhobenen Wertes vom Europäischen Fonds für regionale Entwicklung (EFRE) erfolgen, der die Differenz zwischen dem Sommer- und Winterverbrauch prozentual abbildet. Zum Anderen kann auch auf Basis der vorhandenen Messwerte eine Differenz gebildet werden, wobei hier berücksichtigt werden muss, dass sich sehr lückenhafte Daten, die unter Umständen ganze Monate gar nicht enthalten, nur bedingt für die Konfiguration eignen. 
%-----------------------------------------------------------------------
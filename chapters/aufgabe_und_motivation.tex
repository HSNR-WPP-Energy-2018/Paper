%!TEX root = ../paper.tex

\section{Aufgabe \& Motivation}

Im Laufe dieses Berichts werden mehrere Algorithmen vorgestellt, die sich mit der Interpolation, also dem Finden einer Funktion zu gegebenen, diskreten Daten, von Werten beschäftigt. Der Ursprung basiert dabei auf einer Masterarbeit, die ausrechnet, ob und wann eine Solaranlage sich rentiert. Dabei werden mithilfe von Algorithmen gegebene Werte (also zum Beispiel Anzahl an Solaranlagen, gedeckter Flächeninhalt und Anschaffungskosten) genommen und diese mit den Lastgängen verglichen. Ein Lastgang ist der Energie- / Stromverbrauch über einen bestimmten Zeitraum. Benutzer sollen in der Lage sein anhand ihrer Lastgänge und ihrer Eingabeparameter möglichst genau zu berechnen, wann und ob sich die Anschaffung einer Solaranlage rentiert. Früher war die Rechnung vergleichsweise einfach - da der Erlös von verkauftem Strom höher war als die Verbrauchskosten von Strom hatte man sämtlichen Strom einfach verkaufen können und konnte so ganz einfach den erzeugten Strom mit dem Verkaufspreis multiplizieren. Mittlerweile werden Solaranlagen allerdings nicht mehr so subventioniert wie es früher war - der Verkaufspreis für Strom ist nun deutlich geringer als der Kaufpreis. Daher ist es sinnvoller, erzeugten Strom selbst zu benutzen, statt ihn zu verkaufen, und den erzeugten Strom dementsprechend nur zu verkaufen, wenn man zu der Zeit der Erzeugung keinen Eigenbedarf hat. Um möglichst vielversprechende Aussagen über die Rentabilität treffen zu können, ist es heute allerdings erforderlich, dass eine solche Gegenüberstellung, also dem Verbrauch anhand von Lastgängen und die Erzeugung von Strom, durchgeführt und diese mit den jeweiligen Faktoren multipliziert wird. Je mehr Lastgänge vorhanden sind, desto genauer ist am Ende die Rentabilitätsrechnung, allerdings kommt es selten vor, dass Verbraucher über längere und lückenlose Zeiträume von Lastgängen verfügen. Es ist eher so, dass wenn ein Verbraucher Lastgänge zur Verfügung hat, dann sind diese meistens begrenzt, z. B. nur für ein halbes Jahr. Auch kann es vorkommen, dass dort mehrere Lücken sind, da das Gerät für ein paar Tage ausgefallen ist oder einen technischen Defekt hatte. Die Hauptaufgabe ist also auf der einen Seite die Interpolation von Werten, um genau diese Lücken zu füllen. Auf der anderen Seite, weil sich Solaranlagen wahrscheinlich nicht in solch kurzer Zeit rentieren, ist es die Erstellung von Prognosen, da der Stromverbrauch zeitlich variable ist: Im Winter wird zum Beispiel deutlich mehr verbraucht als im Sommer. Der letzte Fall ist, dass Verbraucher gar keine Lastgänge zur Verfügung haben. Diesen soll es ermöglicht werden deren Lastgänge anhand eines Profilgenerators zu schätzen.

Für das Projekt liegen uns zwei Lastgänge in Form einer Excel Tabelle vor. Eins vom Jahr 2016, das zweite für das Jahr 2017. Jeweils gegeben ist ein Datum, eine Uhrzeit und der gemessene Wert in kWh und das im 15-Minuten-Takt.
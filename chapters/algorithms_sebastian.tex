%!TEX root = ../paper.tex

\subsubsection{Durchschnittsbildung}
Eine Erweiterung zur Interpolation aus Daten des Vortages stellt die Durchschnittsbildung dar. Hierbei fließen zusätzlich zum Wert des Vortages zur selben Uhrzeit weitere Werte in definierten Abständen ein. Der Algorithmus unterstützt verschiedenste Kombinationen, eine Möglichkeit ist der Einfluss von Werten im Wochenabstand und Tagesabstand jeweils zur gleichen Uhrzeit und zusätzlich eine definierte Anzahl an direkt umliegenden Werten. Die Anzahl der Werte pro Intervall ist dabei frei wählbar und gilt jeweils vor und nach dem Zeitpunkt des Interpolationswertes für gleich viele Werte.

Bereits vorhandene Werte werden nicht interpoliert, sondern übernommen. Bei der Bildung des Durchschnittswerts, der das Interpolationsergebnis darstellt, sind verschiedene Ansätze denkbar. Der einfachste Ansatz ist die einfache Mittelwertbildung, bei der die Summe der Werte durch die Anzahl der Werte geteilt wird. In der Praxis sollten bestimmten Werte jedoch einen höheren Einfluss auf das Ergebnis erhalten als andere, weshalb die Berechnung entsprechend angepasst wird.
Zum Einen ist eine Gewichtung in Abhängigkeit der Entfernung zum Interpolationszeitpunkt möglich, so dass die äußersten Werte das Ergebnis deutlich weniger beeinflussen als nah am Zeitpunkt liegende Werte.
Auch der Einfluss der Werte verschiedener Intervalle kann unterschiedlich stark sein. Deshalb ist es möglich, jeder Konfiguration für ein Intervall ein Gewicht zuzuweisen, mit dem die Werte multipliziert werden. Die Bestimmung der Summe an Werten wurde dementsprechend angepasst, so dass die Gewichte das Ergebnis nicht verschieben.

Die größte Herausforderung dieses Ansatzes ist die Konfiguration der Interpolationsintervalle. Eine variable Anzahl an Parametergruppen (eine Gruppe pro Intervall) mit zahlreichen einzelnen Parametern muss definiert und konfiguriert werden. Hierfür können entweder Fachwissen eingesetzt oder verschiedenste Konfigurationen ausprobiert werden.
Bei sehr großen Lücken sind die Ergebnisse je nach Konfiguration nicht mehr allzu gut, da nur wenige vorhandene Werte in die Berechnung einfließen können.

\subsubsection{Datenbankansatz}
Im Gegensatz zu den anderen Algorithmen, die lediglich gegebene Werte des zu interpolierenden Datensatzes in die Berechnung einfließen lassen, interpoliert der Datenbankansatz Werte auf Basis mehrerer anderer Datensätze, deren Mittelwert fehlende Werte bildet.

Während einer Vorverarbeitungsphase, nach dessen Abschluss beliebig viele Werte interpoliert werden können, wird die Datenbank aus zahlreichen gesammelten Lastprofilen gebildet. Eingenschaften, entweder einfache Attribute (bspw. Haus/Wohnung, Heizungstyp) oder numerische Einordnungen (Personenanzahl, Wohnfläche, etc.), klassifizieren die Datensätze und unterteilen diese in verschiedene Gruppen.
Zur Interpolation werden lediglich Lastprofile verwendet, deren Eigenschaften denen des Ergebnisdatensatzes entsprechen. Die geforderten Eigenschaften sind als zusätzliche Eingabeparameter anzugeben.
Aktuell ist die Interpolation nach der Filterung der Datensätze eine einfache Mittelwertbildung der einzelnen Werte für den jeweiligen Zeitpunkt. Hier ist eine Erweiterung zu einem gewichteten Einfluss möglich. Damit können auch Datensätze verwendet werden, die nicht in allen Kriterien den Anforderungen entsprechen. Die damit steigende Ungenauigkeit wird durch eine bessere Glättung aufgrund der gestiegenen Anzahl an Datensätzen wieder ausgeglichen.

Aufgrund von Änderungen in der Zuordnung zwischen Wochentag und Datum innerhalb eines Jahres können die Verbrauchswerte nicht einfach nur mit dem Datum gespeichert werden, ohne dass ein Datensatz nur für ein einzelnes Jahr gültig ist. Stattdessen werden die Datensätze während der Vorverarbeitungsphase in ein anderes Format transformiert. Nach der Umformung werden für definierte Intervalle, entweder Quartale, Monate oder Wochen, Werte für jeden Wochentag und jede Uhrzeit innerhalb des vorgegebenen Abstands gespeichert. Sofern mehrere Werte für ein Intervall, einen Wochentag und eine Uhrzeit vorhanden sind, wird der Mittelwert gebildet und gespeichert. Damit können die Unterschiede im Verbrauch zwischen Wochentagen und dem Wochenende bei der Interpolation berücksichtigt werden.

Im Gegensatz zu den anderen Algorithmen benötigt der Datenbankansatz keinen lückenhaften Datensatz. Die Berechnung kann alleine auf Basis der definierten Eigenschaften erfolgen. Eine Ergänzung der Daten eines lückenhaften Datensatzes ist ebenfalls möglich.

Um möglichst gute Ergebnisse liefern zu können, benötigt der Datenbankansatz eine große Datenbasis, deren Sammlung eine Herausforderung ist. Außerdem erfordert die zusätzliche Kategorisierung zusätzlichen, nur manuell durchführbaren Aufwand.
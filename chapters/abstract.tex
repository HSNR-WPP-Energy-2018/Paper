%!TEX root = ../paper.tex

% \noindent Die Hauptaufgabe des Projektes war die Interpolation von Lastgängen. Der Projektbericht gibt einen Überblick darüber, wie die Realisierung dieser Aufgabe durch das Projektteam gestaltet wurde. Dabei geht er zunächst darauf ein, warum eine Interpolation notwendig ist, und es werden Überlegungen getroffen, wie sich diese bedarfsgerecht realisieren lässt. Anschließend werden mehrere Algorithmen vorgestellt, angefangen von einfachen Verhältnis-Matrizen über lineare und polynomiale Verfahren bis hin zu Datenbankansätzen. Zusätzlich zu den Algorithmen werden Überlegungen zu Heuristiken vorgestellt, die entweder ergänzend zu den Algorithmen oder gar eigenständig Werte generieren können. Danach werden dann alle möglichen Algorithmen und Heuristiken miteinander auf verschiedenen Testdaten verglichen und auch bewertet. Zu guter Letzt folgt ein Ausblick: Die Tests und Bewertungen der Algorithmen, aber auch die Verfahren selbst, können durch die Verfügbarkeit weiterer Lastgänge deutlich verbessert werden. Auch wird auf komplexere Algorithmen und neuronale Netze als Alternative hingewiesen.

\noindent Zur Bewertung der Rentabilität einer Solaranlage ist die Kenntnis der Stromverbräuche für detaillierte zeitliche Intervalle notwendig. In Abhängigkeit der Deckung von Erzeugung und Verbrauch des Stroms ergeben sich unterschiedliche Notwendigkeiten zum Einkauf externen Stroms. Bei der Aufzeichnung des Energieverbrauchs kann es aufgrund zahlreicher Probleme zu Ausfällen kommen.
Das Projekt befasst sich wesentlich mit der Vervollständigung dieser fehlenden Werte. Dabei wurden mathematische Ansätze, unter anderem eine Interpolation nach dem Newton-Verfahren, und wissensbasierte Verfahren, die naheliegende vorhandene Werte oder die Messwerte anderer Datensätze verwenden, eingesetzt. Die Evaluation der Algorithmen ergab gerade bei großen Datenlücken gute Ergebnisse der wissensbasierten Ansätze.
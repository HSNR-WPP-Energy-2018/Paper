%!TEX root = ../paper.tex

Dieser Bericht beinhaltet hauptsächlich Themen zur Interpolation von Lastgängen. Dabei geht er zunächst darauf ein, wozu eine Interpolation notwendig ist und es werden Überlegungen getroffen, wie das am besten möglich ist. Anschließend werden mehrere Algorithmen vorgestellt, angefangen von einfachen Verhältnis-Matrizen, über lineare und polynomiale Verfahren bis hin zu Datenbankansätzen. Zusätzlich zu den Algorithmen werden Überlegungen zu Heuristiken vorgestellt, die entweder ergänzend zu den Algorithmen oder gar eigenständig Werte interpolieren können. Danach werden dann alle möglichen Algorithmen und Heuristiken miteinander auf verschiedene Testdaten verglichen und auch bewertet. Zu guter Letzt folgt ein kleiner Ausblick: die Tests und Bewertungen der Algorithmen, aber auch die Verfahren selbst, können durch mehr Lastgänge deutlich verbessert werden. Auch wird auf komplexere Algorithmen und neuronale Netze als Alternative hingewiesen.